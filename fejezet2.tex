\chapter{Hálózati oprendszer konfigurálása}
\section{Állomásnév}
Az első parancs viszonylag egyszerű, az állomásnevet állíthatjuk. Ez egy jobb hálózaton hasznos, jelenleg pedig a CLIben látjuk a hatását.

Érdemes egyébként célszerű neveket adni: Például \emph{Switch-Floor-1}
   A \verb|hostname <név>| paranccsal lehet állítani.

   Példa:
\begin{parancs}
   \begin{enumerate}
	\item \verb|en| 
	\item \verb|conf t|
	\item \verb|hostname cisco|
   \end{enumerate}
\end{parancs}
   Eredmény:

   \verb|cisco#| fogad majd minket


Visszavonni a \verb|no hostname| paranccsal lehet, a no szó általában amúgy hasznos lesz.

\section{Biztonság}

\subsection{Enable jelszó}
Enable módban sok jogunk van ezért érdemes lejelszavazni. Ezt lehet \emph{sima} vagy \emph{secret} itt a különbség, hogy a \emph{secret} titkosítva tárolja a jelszót.

Egy nem secret jelszó beállítása:
\begin{parancs}
\begin{enumerate}
	\item \verb|en| 
	\item \verb|conf t|
	\item \verb|enable secret <jelszó>|
\end{enumerate}

\end{parancs}
Hogy ezt kipróbálhassuk lépjünk ki a \verb|disable| paranccsal, majd próbáljunk újra belépni!

\subsection{Console és VTY jelszó}
Érdemes beállítani jelszót a konzolos hozzáférésre, és virtuális terminálra is.
Ez sem nehéz:
\begin{parancs}
\begin{enumerate}
	\item \verb|en| 
	\item \verb|conf t|
	\item \verb|line console 0| vagy \verb|line vty 0 15|, attól függően, hogy hol akarjuk. (a számok a portot jelölik: ált. 1 console line van, és 16 vty vonal, de ezek változhetnak)
	\item \verb|password <jelszó>|
	\item \verb|login| Ne felejtsük el!
	\item \verb|exit| (már ha nincs több dolgunk)
\end{enumerate}

\end{parancs} 
Amúgy a 3. pontot érdemes memorizálni, mert még használjuk majd ;)


Fontos hogy ezek amúgy titkosítatlan jelszavak, szóval érdemes ez ellen tenni (Ez csak a konfigurációs állományban levő jelszavakra vonatkozik!)
\begin{parancs}
\begin{enumerate}
	\item \verb|en| 
	\item \verb|conf t|
	\item \verb|service password-encryption|
\end{enumerate}

\end{parancs} 
Ez (meg amúgy sok minden más) látható a \verb|show running-config| vagy \verb|show startup-config| paranccsal.

\section{Banner}
Kevésbé izgalmas téma, de fontos, mert néhol ki \emph{ kell }írni, dolgokat.

A MOTD (Message Of The Day) rögtön loginkor megjelenik.
Ennek beállítása:

(ja meg disclaimer: az \emph{en;conf t}-t mindenki képzelje oda nem írom le többet.)
\begin{parancs}
\verb|banner motd # <üzenet> #|

(a \verb|#| \emph{nem} lesz benne a szövegben, aközé gépelhetsz.)
\end{parancs}

\section{Mentés\&Reset}
Alapvetően 2 config file van: a running és a startup.

Az előbbi az aktuális, míg a másik az induláskor lép életbe. Ha azt akarjuk, hogy ne vesszenek el a változtatásaink le kell menteni őket a startup configba.

Így:
\begin{parancs}
    \verb|copy running-config startup-config|
\end{parancs}

 Ha valamit ordassan elrontunk tudjuk mind2 configot resetelni:
\begin{parancs}
	Ezeket EXEC módban adjuk ki:

	\verb|reload| a runninghoz (itt majd el kell fogadni)

	\verb|erase startup-config| gondolom ez evidens.

	\verb|delete vlan.dat| ez meg switcheken egy hasznos parancs
\end{parancs}

Gépre lementeni meg ugye: rámegyünk puttyal és kimásoljuk a configot (show után)

\section{Címzés}
\subsection{IP címek}
4 darab 0-255 közötti szám

Hatrozik hozzá egy alhálózati(subnet) maszk: Meghatározza, hogy az eszköz a nagy hálózat melyik alhálózatához tartozik. 

Lehet rendelni fizikai porthoz és virtuális interfészhez.

\subsection{Portok és interfészek}
Fizikai és virtuális portok; Alapból 1: \verb|VLAN1|

\subsection{Kapcsolók címzése (IPv4)}
A távoli eléréshez IP-cím és subnet mask kell.

Hogyan konfiguráld:
\begin{parancs}
	\verb|interface vlan 1|- globálisból inteface config módba vált
	\verb|ip address <ip cím> <subnet mask>| IP cím és alhálózati maszk beállítás
	\verb|no shutdown| vagy \verb|no shu| ne felejtsük el bekapcsolni
\end{parancs}

\subsection{Ellenőrzés}
Legegyszerűbben a ping paranccsal tudjuk a legtöbb eszközön ellenőrizni a kapcsolatot. Erről nem szeretnék sokat írni: \verb|ping <ip cím>|

A kapcsoló interfaceit a \verb|show ip interface brief| paranccsal nézhetjük meg.

