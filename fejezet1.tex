\chapter{Használat}
\section{Kapcsolódás az eszközhöz}
\begin{itemize}
\item{Soros port: Fizikailag}
\item{Telnet/SSH: Neten keresztül (SSH az újabb biztonságosabb)}
\item{AUX: Analóg telefonvonal}
\end{itemize}
\section{Módok}
Az alap \emph{User} móddal nem jutunk nagyon sokáig ezért legtöbbször rögtön az
\verb|en| vagy \verb|enable| paranccsal szoktunk indítani, az \emph{EXEC} módba lépéshez, ahol több jogunk van.
Innen gyakran a \verb|conf t| azaz \verb|configure terminal| paranccsal folytatjuk a \emph{globális konfigurációs mód}ba lépéshez.

\begin{info}[Kinézet]
Amúgy könnyen látni, hogy milyen módban vagyunk, attól függően, hogy mi van az eszköz neve mögött. 


Példák Switch nevű eszközön:

\textbf{User mód:} \verb|Switch>|

\textbf{EXEC mód:} \verb|Switch#|

\textbf{Global config mód:} \verb|Switch (config)#|

\end{info}
